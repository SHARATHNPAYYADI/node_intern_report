\chapter{Appendix}
\section{AMR}

AMR's also called as Autonomous mobile robots. It is a robot which acts without the recourse to human control. These mobile robots are majorly used in manufacturing industries. The first autonomous robots environment were known as Elmer and Elsie, which were constructed in the late 1940s by W. Grey Walter. 

The most important feature required for complete physical autonomy is the ability for robot to take care of itself. Some of these robots can be charged by connecting station, few of them have self docking ability. 

Another feature is sensing of environment. The robots should know the place where they exactly located. This is also called as localization. 

\subsection{Autonomous navigation}

For a robot to associate behaviors with a place (localization) requires it to know where it is and to be able to navigate point-to-point.  Such navigation began with wire-guidance in the 1970s and progressed in the early 2000s to beacon-based triangulation. Current commercial robots autonomously navigate based on sensing natural features. At first, autonomous navigation was based on planar sensors, such as laser range-finders, that can only sense at one level. The most advanced systems now fuse information from various sensors for both localization (position) and navigation. Systems such as Motivity can rely on different sensors in different areas, depending upon which provides the most reliable data at the time, and can re-map a building autonomously.

Rather than climb stairs, which requires highly specialized hardware, most indoor robots navigate handicapped-accessible areas, controlling elevators, and electronic doors. With such electronic access-control interfaces, robots can now freely navigate indoors. Autonomously climbing stairs and opening doors manually are topics of research at the current time.

As these indoor techniques continue to develop, vacuuming robots will gain the ability to clean a specific user-specified room or a whole floor. Security robots will be able to cooperatively surround intruders and cut off exits. These advances also bring concomitant protections: robots' internal maps typically permit "forbidden areas" to be defined to prevent robots from autonomously entering certain regions.